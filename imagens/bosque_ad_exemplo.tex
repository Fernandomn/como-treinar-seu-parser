\begin{figure}[!ht]
    \centering
    % \includegraphics{}
    \begin{tabular}{l}
        =>N:art('o' <artd> M P) Os\\
        =H:n('promotor' M P) promotores
    \end{tabular}
    \caption[Exemplo de nós no formato AD]{Exemplo de nós no formato AD. Como descrito em \citeonline{freitas2007biblia}, \textquote{A funçao de \textquote{os} é de dependente de um núcleo nominal (N) que está a sua direita (>), por isso a marcação >N; a forma de \textquote{os} é artigo. Tem-se entao o par de função e forma >N:art. \textquote{promotores}, por sua vez, é o núcleo (H) do sintagma, e sua forma é substantivo/nome (n). O par função e forma é portanto H:n}. Adaptado de \citeonline{freitas2007biblia}}
    \label{fig:bosque_ad_exemplo}
\end{figure}