\begin{figure}[!h]
    \centering
    \begin{minipage}{10cm}
        \begin{tabbing}
            \=(VP \=\+\\
            \>    (IN \=porque)\+\\
            \>        (VP \=\+\\
            \>            (VBP tem)\-\\
            \>        )\\
            \>        (NP \=\+\\
            \>            (NN gente)\-\\
            \>        )\\
            \>        (VP \=\textbf{comprando e}\+\\
            \>            (\textbf{VBG vendendo})))
        \end{tabbing}
    \end{minipage}
    % \includegraphics{}
    \caption[Exemplo de como coordenações \textit{single word} devem se comportar]{Exemplo de como coordenações \textit{single word} devem se comportar. Fragmento da conversão da sentença \textit{CF400-2 Diretor do Banco Central não acredita que o real esteja valorizado porque \guillemotleft tem gente comprando e vendendo\guillemotright}}
    \label{fig:bosque_exemplo_flat}
\end{figure}

