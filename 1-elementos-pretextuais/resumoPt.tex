% resumo em português
\setlength{\absparsep}{18pt} % ajusta o espaçamento dos parágrafos do resumo
\begin{resumo}

    Classificadores sintáticos automatizados (\textit{parsers}) são um recurso computacional estudado há bastante tempo dentro da área de Processamento de Linguagem Natural. Apesar disto, é difícil encontrar \textit{parsers} para a língua portuguesa. 
    % Existem \textit{parsers} baseados em regras, mas os mais conhecidos são os estatísticos, que precisam de um conjunto de dados adaptado, para que seja realizado o seu treinamento.
    Enquanto existem diversos métodos de \textit{parsing} baseados em regras na literatura, atualmente os métodos mais investigados se baseiam em métodos estatísticos, e baseados em dados. Os estatísticos, em particular, necessitam de um conjunto de dados de entrada para serem treinados.
    
     Curiosamente, existe uma quantidade interessante de conjuntos de dados pré-classificados, em português, acessíveis ao público, que podem ser usados para treinar \textit{parsers}.
    
    Propõe-se, então, o treinamento de um \textit{parser} conhecido, desenvolvido para a língua inglesa, com dados de treino da língua portuguesa. Para tal, será realizada a \textit{transdução} dos dados
    % da língua portuguesa para a língua inglesa, sem perda de informação lexical.
    nos seus respectivos formatos originais para o formato aceito pelo \textit{parser} escolhido, sem perda de informação lexical.

 \textbf{Palavras-chaves}: \textit{parsers}, classificador sintático, Língua Portuguesa, transdução.
\end{resumo}