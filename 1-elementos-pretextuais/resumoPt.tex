% resumo em português
\setlength{\absparsep}{18pt} % ajusta o espaçamento dos parágrafos do resumo
\begin{resumo}

    Classificadores sintáticos automatizados (também conhecidos como \textit{parsers}) são um recurso computacional estudado 
    % há bastante tempo dentro da 
    na área de Processamento de Linguagem Natural. São dispositivos capazes de realizar a classificação morfossintática de sentenças escritas em linguagem natural. Apesar de serem estudados há bastante tempo, são poucos os \textit{parsers} disponíveis desenvolvidos para o processamento da língua portuguesa.
    
    % Existem \textit{parsers} baseados em regras, mas os mais conhecidos são os estatísticos, que precisam de um conjunto de dados adaptado, para que seja realizado o seu treinamento.
    Existem diversos métodos de \textit{parsing} baseados em regras pré-definidas na literatura. Porém, atualmente os métodos mais investigados se baseiam em métodos estatísticos.
    % , e baseados em dados. Os estatísticos, em particular, 
    Tais métodos necessitam de um conjunto de dados de entrada pé-classificados para serem treinados. Estes conjuntos de dados são chamados de bancos de árvores (\textit{treebanks}).
    
    Dada a existência prévia de \textit{treebanks} próprios para o processamento da língua portuguesa; e \textit{parsers} com boa performance, mas que não estão adaptados para esta mesma língua, 
    % Curiosamente, existe uma quantidade interessante de conjuntos de dados pré-classificados, em português, acessíveis ao público, que podem ser usados para treinar \textit{parsers}.
    propõe-se, então, o treinamento de um \textit{parser} conhecido, desenvolvido com foco na língua inglesa, com dados de treino da língua portuguesa. Para tal, será realizada a \textit{transdução} dos dados
    % da língua portuguesa para a língua inglesa, sem perda de informação lexical.
    nos seus respectivos formatos originais para o formato aceito pelo \textit{parser} escolhido, sem perda de informação lexical. 
    
    % Para tal, desenvolveu-se uma metodologia de adaptação entre \textit{treebanks}, de modo a gerar uma efetiva correlação entre conjuntos de dados distintos.

 \textbf{Palavras-chaves}: \textit{parsers}, classificador sintático, Língua Portuguesa, transdução.
\end{resumo}