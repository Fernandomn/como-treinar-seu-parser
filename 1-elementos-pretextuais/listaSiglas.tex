% ---
% inserir lista de abreviaturas e siglas
% ---
\begin{siglas}  
    \item [CD] \textit{Corpus} de Destino. \textit{Corpora} cujas estruturas será reproduzido no processo de transdução
    \item[CETEM] \textit{Corpus de Extractos de Textos Electrónicos NILC}, Corpus criados pelo projeto Linguateca. Possui as variantes CETEMFolha e CETEMPublico.
    \item[CFG] \textit{Context Free Grammars}, Gramáticas Livres de Contexto
    \item[CLI] \textit{Command Line Interface}, \textit{softwares} que permitem a interação por meio do terminal de comandos do sistema.
    \item[CO] \textit{Corpus} de Origem. \textit{Corpora} que passarão pelo processo de transdução para se assemelharem a algum CD.
    \item[FAQ] \textit{Frequently Asked Questions}, perguntas mais frequentes
    \item[IA] Inteligência Artificial
    %   \item[OotB] \textit{Out-of-the-Box}, \textquote{fora da caixa}
    \item[PCFG] \textit{Probabilistic Context Free Grammars}, Gramáticas Livres de Contexto Probabilísticas
    \item[PLN] Processamento de Linguagem Natural
    \item[POS] \textit{Part of Speech}, Parte do Discurso
    \item[PTB] \textit{Penn TreeBank}
    %   \item[RNN] \textit{Recursive Neural Networks}, Redes Neurais Recursivas
    %   \item[RNR] Ver RNN
    \item[SP] \textit{Stanford Parser}
    \item[UD] \textit{Universal Dependencies}, projeto para criar \textit{tagsets} unificados para \textit{treebanks} de diversas línguas
  
\end{siglas}