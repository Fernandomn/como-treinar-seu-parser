\begin{longtable}{|p{0.05\linewidth}|p{0.15\linewidth}|p{0.15\linewidth}|p{0.15\linewidth}|p{0.15\linewidth}|p{0.3\linewidth}|}
\caption{Tabela de conversão: CINTIL para PTB}\\
\hline
\textbf{Cat.} & \textbf{Tag Original} & \textbf{Nome da Tag} & \textbf{Tag Convertida} & \textbf{Ocorrências} & \textbf{Observações}\\
\hline
\endfirsthead
\multicolumn{5}{c}%
{\tablename\ \thetable\ -- \textit{Continuação da página anterior}} \\
\hline
\textbf{Cat.} & \textbf{Tag Original} & \textbf{Nome da Tag} & \textbf{Tag Convertida} & \textbf{Ocorrências} & \textbf{Observações} \\
\hline
\endhead
\hline \multicolumn{5}{r}{\textit{Continua na próxima página}} \\
\endfoot
\hline
\endlastfoot
    1 & A & Adjetivo & JJ & 5527 & \\
    1 & A' & Sintagma Adjetival & ADJP & 114 & \\
    1 & ADV & Advérbios & RB & 5510 & \\
    1 & ADV' & Sintagma Adverbial & ADVP & 912 & \\
    1 & ADVP & Sintagma Adverbial & ADVP & 428 & \\
    1 & AP & Sintagma Adjetival & ADJP & 1456 & \\
    1 & ART & Artigo & DT & 15583 & \\
    2 & ART' & Artigo & NP & 1 & Equivaleria ao constituinte interemediário do \textit{Determiner Phrase} (DP) \citeonline{mioto2013novo}. Porém, PTB não prevê esse tipo de estrutura. O sintagma mais indicado para receber determinantes (artigos) foi, portanto, NP\\
    3 & C & Complementador & CC & 275 & Será explicado na sessão \ref{subsec:cintil-c}\\
    3 & C' & Sintagma Complemental & \_CP\_ & 2 & Será explicado na sessão \ref{subsec:cintil-c}\\
    2 & CARD & Cardinais & CD & 2028 & Números cardinais\\
    2 & CARD' & Sintagmas Cardinais & NP & 504 & \cite{bracketing_ptb} prevê que conjuntos de números são marcados como NP\\
    2 & CL & Clíticos & PRP & 717 & No CINTIL, ocorre apenas como pronome. De acordo com \citeonline{cintil_handbook}, \textquote{\textit{A clitic pronoun has category CL. It is the head of an NP.}}\\
    3 & CONJ & Conjunções & CC & 2460 & Será explicado na sessão \ref{subsec:cintil-conj}\\
    3 & CONJ' & Sintagma Conjuntivo & \_CONJP\_ & 92 & Será explicado na sessão \ref{subsec:cintil-conj}\\
    3 & CONJP & Sintagma Conjuntivo & CONJP & 609 & Será explicado na sessão \ref{subsec:cintil-conj}\\
    3 & CP & Sintagma Complemental & SBAR & 1434 & Será explicado na sessão \ref{subsec:cintil-c}\\
    1 & D & Artigo & DT & 29 & \\
    2 & D1 & Quantificadores & DT & 1 & Não ocorre no Handbook, só no site. Único caso em que essa \textit{tag} aparece, D1 se comporta como Artigo\\
    2 & D2 & Quantificadores & JJ & 1 & Não ocorre no Handbook, só no site. Único caso em que essa \textit{tag} aparece, D2 se comporta como Adjetivo\\
    2 & DEM & Demonstrativos & DT & 1013 & Para o PTB, \textit{this}, \textit{that}, \textit{these}, \textit{those} são, também, determinantes. Logo, DT\\
    1 & ITJ & Interjeições & UH & 4 & \\
    2 & ITJ' & Sintagma de Interjeição & INTJ & 4 & Por \cite{bracketing_ptb}, \textquote{Interjeição. Corresponde aprocimadamente à etiqueta morfossintática UH (veja as diretrizes [\citeonline{posPTBguidelines}]}\footnote{\textquote{\textit{INTJ | Interjection. Corresponds approximately to the part-of-speech tag UH (see the POS guidelines [Santorini 1990]).}}. Tradução própria.}\\
    1 & N & Substantivo & NNS & 32989 & \\
    1 & N' & Sintagmas Nominais & NP & 18043 & \\
    1 & NP & Sintagmas Nominais & NP & 32258 & \\
    2 & ORD & Ordinais & CD & 378 & PTB não prevê o uso de ordinais. Ou melhor: eles costumam ser postos em locuções nominais.\\% Não é uma boa transdução\\
    1 & P & Preposição & IN & 13920 & \\
    2 & P' & Sintagmas Preposicionais & PP & 337 & Não ocorre no Handbook\\
    2 & PERCENT & simbolo percentual & NN & 164 & Nota 1: pode ser pronome + substantivo também (\textquote{por cento}). Nota 2: PTB considera o \% como NN (\textit{single noum})\\
    2 & PERCENT' & Sintagma percentual & NP & 80 & PTB considera como NP\\
    2 & PERCENTP & Sintagma percentual & NP & 36 & \\
    3 & PNT & Pontuação & ? & 14748 & Explicado na sessão \ref{subsec:cintil-pnt}\\
    2 & POSS & Possessivos & PP\$ & 620 & \\
    2 & POSS' & Possessivos & NP & 10 & Não existe um sintagma pronominal no PTB. Mantivemos como NP\\
    1 & PP & Sintagmas Preposicionais & PP & 15382 & \\
    2 & PRS & Pronomes Pessoais & PRP & 395 & \\
    2 & QNT & Quantificadores & PRP & 889 & De acordo com (\citeonline[p~55]{Castilho2010gramatica}), \textquote{Os pronomes abrigam as seguintes subclasses [...]: pessoais, demonstrativos, possessivos e quantificadores [...]}\\
    2 & QNT' & Sintagma de Quantificadores & NP & 19 & Como supracitado, se refere ao sintagma que abriga quantificadores (pronomes)\\
    2 & REL & Relativos & PRP & 861 & Pronomes relativos\\
    1 & S & Sentença & S & 24393 & \\
    1 & V & Verbos  & VB & 13281 & \\
    1 & V' & Sintagma Verbal & VP & 2745 &  \\
    1 & VP & Sintagma Verbal & VP & 15284 & 
\label{tab:tab_cintil}
\end{longtable}