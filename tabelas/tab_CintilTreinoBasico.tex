\begin{table}[h!]
    \centering
    \begin{tabular}{p{0.8\textwidth}}
        \begin{itemize}
            \item [-cp] \textit{ClassPath}. Indica o diretório onde se encontra a classe principal a ser executada
            \item [-mx4g] Quantidade de memória usada. No caso, 4 GB.
            \item [LexicalizedParser] \textit{Parser} utilizado, dentre os disponibilizados
            \item [-train] Treino. Logo em seguida, um diretório e a lista de arquivos a serem usados para treinar
            \item [-saveToSerializedFile] Salva o resultado do treino num arquivo binário, cujo diretório está indicado na sequência
            \item [-saveToTextFile] Salva o resultado do treino num arquivo de texto, cujo diretório está indicado na sequência
        \end{itemize}
    \end{tabular}
    \caption[Comandos para um treino simples do \textit{Stanford Parser}]{Comandos para um treino simples do \textit{Stanford Parser}, utilizando o terminal.}
    \label{tab:tab_treino_basico_cintil}
\end{table}