\begin{table}[h!]
    \centering
    \begin{tabular}{p{0.8\textwidth}}
        \begin{itemize}
            \item [-cp] \textit{ClassPath}. Indica o diretório onde se encontra a classe principal a ser executada
            \item [-mx4g] Quantidade de memória usada. No caso, 4 GB.
            \item [LexicalizedParser] \textit{Parser} utilizado, dentre os disponibilizados
            \item [-writeOutputFiles ] Indica que os testes imprimirão arquivos de saída, a serem definidos
            \item [-outputFilesDirectory] Define o diretório onde os arquivos de saída serão escritos. 
            \item [-loadFromSerializedFile] Carrega a gramática serializada, gerada na execução de treinamento anterior
            \item [-testTreebank] Diretório onde se encontra o treebank a ser usado para teste. Os números no formato $a-b$ indicam o primeiro e o último arquivo, respectivamente. Números no formato $a-b,c-d$ indicam dois blocos de arquivos. Atente para não usar o mesmo bloco dos treinos, ou o parser passará por \textit{overfitting}, e terá resultados enviesados.
        \end{itemize}
    \end{tabular}
    \caption[Comandos para um teste simples do Stanford Parser]{Comandos para um teste simples do Stanford Parser, utilizando o terminal.}
    \label{tab:tab_teste_basico_cintil}
\end{table}