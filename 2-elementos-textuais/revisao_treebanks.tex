
\section{\textit{Treebanks}}
\label{treebank}

Em Processamento de Linguagem Natural, não é possível avaliar textos em seu contexto original o tempo todo. Ao invés disso, faz-se uma coleção de textos, que servirão como amostra para a análise. Esse corpo de textos é chamado de \textit{corpus}. Um conjunto de corpus é chamado \textit{corpora}. Como destacado em \cite[p~6]{Manning1999FoundationsNLP},
\begin{displayquote}
    \textquote{\textit{Adopting such a corpus-based approach, people have pointed to the earlier advocacy of empiricist ideas by the British linguist J.R. Firth, who coined the slogan \textquote{You shall know a word by the company it keeps}}}
    \footnote{\textquote{Adotando tal abordagem baseada em corpus, pessoas apontaram para as primeiras defesas das ideias empiricistas pelo linguista britânico] J. R. Flirth, que forjou o slogan \textquote{Você deve conhecer uma palavra pela companhia que esta mantém}}. Tradução própria.}.
\end{displayquote}

Seu uso não é novo. Pelo contrário, em 1951, Zelling Harris tenta descobrir procedimentos no qual a estrutura de uma linguagem possa ser encontrado automaticamente. Como destacado em  (\textit{\Ibidem[p~6]{Manning1999FoundationsNLP}}),
\begin{displayquote}
    \textquote{\textit{While this work had no thoughts to computer implementation, and is perhaps somewhat computationally naive, we find here also the idea that a good grammatical description is one that provides a compact representation of a corpus of texts}}
    \footnote{\textquote{Enquanto este trabalho não pensava numa implementação computacional, e é de certa forma ingênuo computacionalmente, encontramos aqui também a ideia de que uma boa descrição gramatical é uma que provenha de uma representação compacta de um corpus de textos}. Tradução própria.}.
\end{displayquote}

Diversas técnicas de \textit{parsing} utilizam de aprendizado supervisionado. Portanto, precisamos de uma fonte de dados que sirvam para o treino e para o teste destes sistemas. No nosso caso, usamos bancos de árvores, ou \textit{treebanks}. \textit{Treebanks} são, como descrito em  \cite[p~412]{Manning1999FoundationsNLP}, \textquote{\textit{some examples of the kinds of parse trees that are wanted. A collection of such example parses is referred to as a treebank}}
\footnote{\textquote{Alguns exemplos dos tipo de análise em árvore desejados. Uma coleção de tais árvores de exemplo são denominados \textit{treebank}}. Tradução própria.}.

\citeonline[p~142]{bick2008FlorestaSintatica} resumem como:
\begin{quote}
    \textquote{Uma floresta sintática
    \footnote{Ao longo do trabalho usaremos os termos \textquote{banco de árvores}, ou \textquote{treebanks}}
    – tradução do inglês \textit{treebank} – é um conjunto de itens (frases) analisados sintaticamente. A cada frase é atribuída uma estrutura sintática hierárquica, e por isso uma frase (sintaticamente analisada) pode ser vista como uma árvore, donde uma floresta nada mais é que um conjunto de frases analisadas sintaticamente e com informação relativa aos níveis de constituintes. Florestas sintáticas costumam ser utilizadas, de maneira geral, tanto em estudos da língua baseados em corpus como no treino de analisadores sintáticos}.
\end{quote}

Ou seja, \textit{treebanks} são \textit{corpora} de sentenças pré-analisadas sintaticamente, de maneira automatizada, semi-automatizada, ou cuja análise foi totalmente feita por humanos.

% É necessário entender as estruturas internas de um \textit{treebank}. Comentaremos a respeito das \textit{POS tags} na sessão \ref{subsec:POStags}. Abordaremos estrutura frasal na sessão \ref{subsec:estrFrasal}. Por fim, 
Uma grande quantidade de grupos criou seus próprios \textit{treebanks} ao longo da história. \citeonline[p~412]{Manning1999FoundationsNLP} destacam que o mais utilizado, refletindo seu tamanho (robustez) e legibilidade, é o \textit{Penn Treebank}. Este, também, é o  \textit{treebank} que usaremos como referência, e o descreveremos na sessão \ref{subsec:PennTB}.

% -------------------------------------------------------------

\subsection{\textit{Penn Treebank}}
\label{subsec:PennTB}

Vários \textit{treebanks} foram construídos ao longo da história\footnote{\citeonline[p~314]{buildingPTB} citam, por exemplo, Lancaster-Oslo / Bergen, o Lancaster UCREL, e o London-Lund Corpus of Spoken English}. Dentre eles, um que recebeu destaque foi o Brown Corpus. \cite[p~314]{buildingPTB} frisa:
\begin{quote}
    \textquote{\textit{The POS tagsets used to annotate large corpora in the past have traditionally been fairly extensive. The pioneering Brown Corpus distinguishes 87 simple tags and allows the formation of compound tags.}}
    \footnote{\textquote{O conjunto de etiquetas morfossintáticas usados para anotar grandes corpora no passado são, tradicionalmente, bem extensivos. O pioneiro, Brown Corpus, distingue 87 etiquetas simples permite a formatação de tags compostas}. Tradução própria}
\end{quote}

A ideia do \textit{Penn Treebank} (PTB) é, indo na contramão, fazer um \textit{corpus} com o \textit{tagset} simplificado. Algumas estratégias foram tomadas para que essa redução fosse possível, uma vez que, é preciso lembrar, a extensa quantidade de \textit{tags} tem razão para acontecer. É uma forma de alcançar \apudonline[p~314]{garside1988computational}{buildingPTB}
\textquote{\textit{the ideal of providing distinct codings for all classes of words having distinct grammatical behaviour}}.
\footnote{\textquote{O ideal de prover códigos distintos para todas as classes de palavras possuindo comportamentos gramaticais distintos}. Tradução própria.}
Uma das primeiras estratégias citadas foi reduzir a redundância de \textit{tags} cuja distinção pode ser obtida pelo léxico da palavra. Por exemplo, (\textit{\Ibidem[p~314]{buildingPTB}})
\textquote{\textit{The Brown Corpus [\ldots] distinguishes three forms of do--the base form (DO), the past tense (DOD), and the third person singular present (DOZ)}}
\footnote{\textquote{O Brown Corpus [\ldots] distingue tres formas de \textquote{do} - a forma base (DO), o tempo passado (DOD), e a terceira pessoa do presente do singular (DOZ)}. Tradução própria}.
Todas estas diferenças podem ser capturadas lexicalmente no momento da análise.

Além da recuperabilidade lexical, também houve a eliminação de \textit{POS tags} cuja distinção é recuperável com referência à estrutura sintática. (\textit{\Ibidem[p~315]{buildingPTB}}),
\begin{quote}
    \textquote{\textit{For instance, the Penn Treebank tagset does not distinguish subject pronouns from object pronouns even in cases where the distinction is not recoverable from the pronoun's form, as with you, since the distinction is recoverable on the basis of the pronoun's position in the parse tree in the parsed version of the corpus.}}
    \footnote{\textquote{Por exemplo, o conjunto de etiquetas do Penn Treebank não distingue pronomes sujeitos de pronomes objeto mesmo em casos onde a distinção não é recuperável pela forma pronominal, tal como \textquote{\textit{you}}, uma vez que a distinção é recuperável na base a posição do pronome na árvore de derivação na versão classificada do corpus}. Tradução própria.}
\end{quote}

Tomar essa medida não só minimiza redundâncias, como também aumenta a consistência do corpus, uma vez que um número reduzido de \textit{tags} reduz a possibilidade de erros / inconsistências.

O PTB foca, também, em marcar a palavra de acordo com sua característica sintática. Por exemplo, a palavra \textit{One}. \cite[p~315-316]{buildingPTB}
\begin{displayquote}
    \textquote{\textit{For instance, in the phrase the one, one is always tagged as CD (cardinal number), whereas in the corresponding plural phrase the ones, ones is always tagged as NNS (plural common noun), despite the parallel function of one and ones as heads of the noun phrase.\\
    By contrast, [\ldots], we encode a word's syntactic function in its POS tag whenever possible. Thus, one is tagged as NN (singular common noun) rather than as CD [\ldots] when it is the head of a noun phrase.}}
    \footnote{\textquote{Por exemplo, no sintagma \textquote{the one}, \textquote{one} sempre é marcado como CD (número cardinal), enquanto que no sintagma plural correspondente \textquote{the ones}, \textquote{ones} é sempre marcado como NNS (substantivo comum plural), independente da função paralela de \textquote{one} e \textquote{ones} como núcleos do sintagma nominal.\\
    Por contraste, [\ldots], nós codificamos uma função sintática de palavra no seu \textit{POS tag} sempre que possível. Portanto, \textquote{\textit{one}} é marcado como NN (substantivo comum singular) ao invés de CD [\ldots] quando é núcleo de um sintagma nominal}. Tradução própria.}
\end{displayquote}

Um diferencial entre o PTB e boa parte dos \textit{treebanks} existentes é a questão da indeterminação. Quando há tanto a ambiguidade de \textit{POS} no texto, como incerteza do anotador\footnote{Anotador é o sistema que atribui \textit{POS tags} a cada palavra}.  Em diversos momentos, o contexto linguístico é capaz de resolver tais diferenças. Nem sempre é possível atribuir uma única \textit{tag} a uma palavra. Para resolver isto, o PTB possibilita ao anotador que atribua mais de uma \textit{tag} a uma palavra, se necessário. Existe a liberdade de atribuir quantas \textit{tags} forem necessárias,
\textquote{\textit{but in practice, multiple tags are restricted to a small number of recurring two-tag combinations}}
\footnote{\textquote{mas na prática, multiplas tags se restringem a um pequeno número de combinações de duas \textit{tags} recorrentes}, como visto em \cite{buildingPTB}. Tradução própria.}.
Para fazer a anotação do PTB, foi usado um processo em duas partes: primeiro, automatizado, e depois um revisão manual.

Por fim, temos a tabela \ref{tab:tags_ptb}, de \textit{POS tags} relativas ao PTB.
\begin{center}
\begin{longtable}{|p{0.1\linewidth}|p{0.3\linewidth}|p{0.4\linewidth}|}
\caption[Tabela de POS tags do Penn Treebank]{Tabela de \textit{POS tags} do \textit{Penn Treebank}, com anotações. Adaptada de \citeonline{santorini1990part2ndprint}}\\
\hline
\textbf{Tag} & \textbf{Legenda original} & \textbf{Tradução da legenda}\\
\hline
\endfirsthead
\multicolumn{3}{c}{\tablename\ \thetable\ -- \textit{Continuação da página anterior}} \\
\hline
\textbf{Tag} & \textbf{Legenda original} & \textbf{Tradução da legenda}\\
\hline
\endhead
\hline 
\multicolumn{3}{r}{\textit{Continua na próxima página}} \\
\endfoot
\hline
\endlastfoot

% \begin{table}[h!]
%     \centering
%     \begin{tabular}{|l|l|l|}
        % Tag & Legenda original & Tradução do nome\\
        \hline
        CC & Coordinating conjunction & Conjunção coordenada\\
        CD & Cardinal number & Número cardinal\\
        DT & Determiner & Determinante/artigo\\
        EX & Existential there & There existencial\\
        FW & Foreign word & Palavra estrangeira\\
        IN & Preposition / subordinating conjunction & Preposição / conjunção subordinada\\
        JJ & Adjective & Adjetivo\\
        JJR & Adjective, comparative & Adjetivo, comparativo\\
        JJS & Adjective, superlative & Adjetivo, superlativo\\
        LS & List item marker & Marcador de item de lista\\
        MD & Modal & Modal\\
        NN & Noun, singular or mass & Substantivo, singular ou conjunto\\
        NNS & Noun, plural & Substantivo, plural\\
        NNP & Proper noun, singular & Substantivo próprio, singular\\
        NNPS & Proper noun, plural & Substantivo próprio, plural\\
        PDT & Predeterminer & Predeterminante\\
        POS & Possessive ending & Encerramento possessivo ('s)\\
        PRP & Personal pronoun & Pronome pessoal\\
        PRP\$ & Possessive pronoun & Pronome possessivo\\
        RB & Adverb & Advérbio\\
        RBR & Adverb, comparative & Advérbio comparativo\\
        RBS & Adverb, superlative & Advérbio superlativo\\
        RP & Particle & Particula\\
        SYM & Symbol (mathematical or scientific) & Símbolo (matematico ou científico)\\
        TO & to & to (para)\\
        UH & Interjection & Interjeição\\
        VB & Verb, base form & Verbo, infinitivo\\
        VBD & Verb, past tense & Verbo, passado\\
        VBG & Verb, gerund/present participle & Verbo, gerûndio / presente particípio\\
        VBN & Verb, past participle & Verbo, passado particípio\\
        VBP & Verb, non-3rd ps. sing. present & Verbo, presente singular não-3ª pessoa\\
        VBZ & Verb, 3rd ps. sing. present & Verbo, presente singular 3ª pessoa\\
        WDT & wh-determiner & Determinante com WH (What, Which)\\
        WP & wh-pronoun & Pronome com WH (who, whose, which, what)\\
        WP\$ & Possessive wh-pronoun & Pronome possessivo com WH (whose)\\
        WRB & wh-adverb & Advérbio com WH (when, where, how, why)
    % \end{tabular}
    % \csvautotabular{tabelas/tabelas de conversão - PTB POS tags.csv}
\end{longtable}
    % \caption{Tabela de POS tags, com anotações. Adaptada de \citeonline{santorini1990part2ndprint}}
    \label{tab:tags_ptb}
\end{center}

\subsection{\textit{Treebanks} para Língua Portuguesa}
\label{subsec:tbPortugues}
Descreveremos nessa seção alguns dos \textit{treebanks} existentes para a língua portuguesa. Estes são os \textit{treebanks} que usaremos no nosso processo de transdução.

\subsubsection{FLORESTA SINTÁ(C)TICA}
\label{subsubsec:florestasintatica}

A Floresta Sintá(c)tica é um projeto colaborativo entre o Linguateca\footnote{https://www.linguateca.pt/} e o VISL\footnote{https://visl.sdu.dk/}. Consta de um conjunto de diversos \textit{treebanks}, em diversos estágios de construção, e diversos usos. O projeto, atualmente com 4 partes distintas. Como visto em \cite{linguatecaFloresta}:
\begin{quote}
    \textquote{Atualmente, a Floresta Sintá(c)tica tem quatro partes, que diferem quanto ao gênero textual, quanto ao modo (escrito vs falado) e quanto ao grau de revisão linguística: o Bosque, totalmente revisto por linguistas; a Selva, parcialmente revista, a Floresta Virgem e a Amazônia, não revistos. Junto, todo esse material soma cerca de 261 mil frases (6,7 milhões de palavras) sintaticamente analisadas.}
\end{quote}

Neste trabalho, utilizamos o Bosque. Ele está disponível no site da Linguateca, em sua versão 8.0, que data de 2008. 

Cabe notar que o Bosque está, atualmente, no projeto Universal Dependencies (UD). Decidimos manter o uso da versão 8.0 por ser disponibilizado, também, em formato com PTB, o que facilitaria (em tese) nosso estudo. O formato disponibilizado pelo UD segue o padrão CoNLL \cite{nivre2007conll}.
\begin{center}
\begin{figure}[!h]
    \centering
    % \includegraphics{}
    % \begin{flushleft}
    \begin{tabular}{l}
        informação textual\\
        nº frase: texto\\
        A1\\
        NÓ RAIZ<\\
        =NÓ 1\\
        ==NÓ 1.1.\\
        ===NÓ 1.1.1\\
        ====NÓ 1.1.1....n\\
        ==NÓ 1.2.\\
        ===NÓ 1.2.1.\\
        ====NÓ 1.2.1....\\
    \end{tabular}
    % \end{flushleft}
    \caption[Formato Árvores Deitadas]{Formato Árvores Deitadas. Adaptado de \citeonline[p~6]{afonso2006arvores}}
    \label{fig:bosque_ad}
\end{figure}
\end{center}
O Bosque segue o formato chamado Árvores Deitadas \cite[p~6]{afonso2006arvores}, que se apresenta como na Figura \ref{fig:bosque_ad}. Como descrito em \cite{freitas2007biblia}, 
\begin{quote}
    \textquote{A cada frase está associada informação textual, isto é, informação relativa ao extracto a que a frase pertence, o número da frase no Bosque e o texto (frase per se). Cada frase é iniciada por A1 (análise 1 da frase). A mesma árvore pode ter mais do que uma análise distinta que são indicadas por A2, A3, etc. [\ldots]\\
    NÓ RAIZ é o nó mais alto da árvore correspondente à sua raiz, por isso é único, isto é, não existem mais nós ao mesmo nível. Assim, o nó raiz não exibe descontinuidade nem pode estar coordenado.\\
    Todos os outros nós constituintes da árvore (NÓ 1 e nós dependentes e os dependentes dos nós dependentes (NÓ 1.1. ou NÓ 1.2. a NÓ 1.1.1....N ou NÓ 1.2.1....N) estão por isso abaixo da raiz da árvore.}
\end{quote}
\begin{center}
\begin{figure}[!ht]
    \centering
    % \includegraphics{}
    \begin{tabular}{l}
        =>N:art('o' <artd> M P) Os\\
        =H:n('promotor' M P) promotores
    \end{tabular}
    \caption[Exemplo de nós no formato AD]{Exemplo de nós no formato AD. Como descrito em \citeonline{freitas2007biblia}, \textquote{A funçao de \textquote{os} é de dependente de um núcleo nominal (N) que está a sua direita (>), por isso a marcação >N; a forma de \textquote{os} é artigo. Tem-se entao o par de função e forma >N:art. \textquote{promotores}, por sua vez, é o núcleo (H) do sintagma, e sua forma é substantivo/nome (n). O par função e forma é portanto H:n}. Adaptado de \citeonline{freitas2007biblia}}
    \label{fig:bosque_ad_exemplo}
\end{figure}
\end{center}
Cabe notar que cada nó segue o formato (F:f), ou seja, \textbf{F}unção e \textbf{f}orma. Como descrito em \cite{freitas2007biblia},
\begin{quote}
    \textquote{A função corresponde à função sintáctica (sujeito, predicador, etc.) que cada constituinte possui em cada oração ou sintagma que compõe a frase. A forma corresponde à estrutura interna dos constituintes, isto é, sintagmas e orações para os nós não terminais, e, para os nós terminais, é usada uma classificação muito próxima das classes de PoS (advérbio, adjectivo, etc.).}
\end{quote}
    
Como podemos ver na Figura \ref{fig:bosque_ad_exemplo}, cada nó é rico em informações morfossintáticas. O tratamento para isto será melhor descrito em \ref{sec:treinando_sp_bosque}. As \textit{tags} utilizadas no Bosque, bem como seu nome, podem ser vistos na Tabela \ref{tab:tab_bosque}, ou no anexo 1 da Bíblia Florestal \cite{freitas2007biblia}. A Bíblia é o manual de anotação, que descreve toda estrutura do Projeto
% Caso deseje, uma versão impressa, 
\cite{afonso2006arvores} 
% foi disponibilizada. Ao leitor curioso, recomendamos a sua leitura. 
.


\subsubsection{CINTIL}
\label{subsubsec:cintil}
O CINTIL é um \textit{dataset} desenvolvido pela \textit{Natural Language and Speech Group} (NLX-Group\footnote{ \url{http://nlx.di.fc.ul.pt/}}) da Universidade de Lisboa\footnote{\url{https://www.ulisboa.pt}}. Seu objetivo é permitir o estudo linguístico da língua portuguesa, variante europeia.

Por \cite[p~1]{narrativeDescriptionCintil}, ele é um corpus de árvores sintáticas de constituência, de textos em português, constituído por 10039 sentenças e 110166 tokens, tirados de diversas fontes de domínios: notícias (8861 sentenças, 101430 tokens), romances (339 sentenças, 3082 tokens). Além disso, há também 779 sentenças (5654 tokens) que são usadas para testes de regressão de gramáticas computacionais que apoiaram a anotação. A Tabela \ref{tab:cintil_tags} demonstra a distribuição do corpus.

\begin{center}
\begin{table}[!ht]
    \centering
    \begin{tabular}{|c|c|c|c|c|}
    \hline
        Sub-corpus & id & Sentences & Tokens & Domain\\
    \hline
        Sentences for regression testing & aTSTS & 779 & 5654 & Test\\
        \hline
        CINTIL-International Corpus of Portuguese & bCINT & 1219 & 13516 & News\\
         & cCINT & 399 & 3082 & Novels\\
         \hline
        CETEMPublico & eCTMP & 7541 & 86905 & News\\
        \hline
        Penn TreeBank (translation) & dPENN & 101 & 1012 & News\\
        \hline
        Total &  & 10039 & 110166 & \\
    \hline
    
    \end{tabular}
    \caption[Distribuição de sentenças e tags pelo CINTIL]{Distribuição de sentenças e tags pelo CINTIL. Adaptado de \citeonline{narrativeDescriptionCintil}}
    \label{tab:cintil_tags}
\end{table}
\end{center}

Foi usada uma classificação semi-automática na criação do \textit{treebank}. Como visto no Iness\footnote{\url{http://clarino.uib.no/iness/page?page-id=port-descr}} \cite{rosen2012open}, num primeiro momento, uma \textit{deep computational grammar} \cite{lxgram} é usada para gerar todas as possíveis árvores em uma sentença. Na sequência, é feita uma desambiguação manual, no qual dois anotadores escolhem a melhor árvore. Em caso de empate, um terceiro especialista servirá de árbitro.

O CINTIL é distribuído em formato XML, e as árvores classificadas tem formato semelhante ao PTB, fazendo separações usando parênteses, e com classes muito parecidas. A Tabela \ref{tab:tab_cintil} demonstra as \textit{tags} originais, e a frequência de uso no banco.

Os padrões de classificação do CINTIL atual está catalogado no \textit{CINTIL TreeBank handbook} \cite{cintil_handbook}.

O CINTIL tem uma versão de 2005, e uma mais atual distribuída em 2012, pelo Metashare
% \footnote{\url{http://metashare.elda.org/repository/browse/cintil-treebank/2a17d622abcd11e1a404080027e73ea242399e2114844f63896f2f92dd31233e/}}.
\footnote{\url{http://shorturl.at/yCGIO}}
Existe o site de divulgação oficial, cintil.ul.pt\footnote{\url{http://cintil.ul.pt/pt/cintilwhatsin.html}}, porém ele não é atualizado há algum tempo. Isso se nota pois o \textit{tagset} disponibilizado por ele está desatualizado, sendo referente à versão anterior. O mais atual segue as diretrizes do supracitado \textit{Handbook}.

