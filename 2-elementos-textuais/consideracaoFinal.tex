% ---
% Conclusão (outro exemplo de capítulo sem numeração e presente no sumário)
% ---
\chapter{Considerações Finais}
\label{chap:considFinal}
Neste trabalho, estudamos \textit{parsers}, suas limitações e possibilidades. Vimos que existem \textit{parsers} principalmente para a língua inglesa. Que há poucos para a língua portuguesa. Que existem conjuntos de dados em Português que podem ser usados para treinar \textit{parsers}. Propusemos, então, a transdução de dados estruturados em Português para o Inglês, sem perda de informação léxica, e realizamos tal transdução. Verificamos que, apesar de ser uma abordagem interessante, o trabalho em questão ainda não é uma solução, quando o problema é a baixa oferta de \textit{parsers} para a língua portuguesa.

Por fim, ficam algumas propostas de trabalhos futuros. Que conste:
A revisão deste trabalho, ampliando os casos tratados (por exemplo, quais situações são descritas não por \textit{POS tags}, mas pela \textit{estrutura} da árvore?);
a correção de estruturas que, para este trabalho, não estão em padrão aceitável, como sinais de pontuação, e a porcentagem da transdução do Bosque (\ref{subsec:percent});
o desenvolvimento de transdutores como alternativa ao desenvolvimento de \textit{parsers};
a implementação de \textit{parser}, com desenvolvimento \textit{out-of-the-box} para o português; 
dentre outras possibilidades. 