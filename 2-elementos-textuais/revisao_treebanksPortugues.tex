\subsection{\textit{Treebanks} para Língua Portuguesa}
\label{sec:tbPortugues}
Descreveremos nessa seção alguns dos \textit{treebanks} existentes para a língua portuguesa. Estes são os \textit{treebanks} que usaremos no nosso processo de transdução.

\subsubsection{FLORESTA SINTÁ(C)TICA}
\label{subsec:florestasintatica}

A Floresta Sintá(c)tica é um projeto colaborativo entre o Linguateca\footnote{https://www.linguateca.pt/} e o VISL\footnote{https://visl.sdu.dk/}. Consta de um conjunto de diversos \textit{treebanks}, em diversos estágios de construção, e diversos usos. O projeto conta atualmente com 4 partes distintas, separadas.

Como visto em \citeonline{linguatecaFloresta}:
\begin{quote}
    \textquote{Atualmente, a Floresta Sintá(c)tica tem quatro partes, que diferem quanto ao gênero textual, quanto ao modo (escrito vs falado) e quanto ao grau de revisão linguística: o Bosque, totalmente revisto por linguistas; a Selva, parcialmente revista, a Floresta Virgem e a Amazônia, não revistos. Junto, todo esse material soma cerca de 261 mil frases (6,7 milhões de palavras) sintaticamente analisadas.}
\end{quote}

Neste trabalho, utilizamos o Bosque. Ele está disponível no site da Linguateca, em sua versão 8.0, que data de 2008. 

Cabe notar que o Bosque está, atualmente, no projeto Universal Dependencies (UD). Decidimos manter o uso da versão 8.0 por ser disponibilizado, também, em formato com PTB, o que facilitaria (em tese) nosso estudo. O formato disponibilizado pelo UD segue o padrão CoNLL (\citeonline{nivre2007conll}).
\begin{center}
\begin{figure}[!h]
    \centering
    % \includegraphics{}
    % \begin{flushleft}
    \begin{minipage}{0.3\textwidth}
        % \begin{tabular}{l}
        % \centering
        
            informação textual\\
            nº frase: texto\\
            A1\\
            NÓ RAIZ<\\
            =NÓ 1\\
            ==NÓ 1.1.\\
            ===NÓ 1.1.1\\
            ====NÓ 1.1.1....n\\
            ==NÓ 1.2.\\
            ===NÓ 1.2.1.\\
            ====NÓ 1.2.1....\\
        % \end{tabular}
    \end{minipage}
    % \begin{tabular}{l}
        
    % \end{tabular}
    
    % \begin{minipage}{0.7\textwidth}
    %     \centering
    %     \begin{tabbing}
    %         (S \=\+\\
    %             (NÓ \=1\+\\
    %                 (NÓ \=1.1\+\\ 
    %                     (NÓ \=1.1.1\+\\ 
    %                         (NÓ 1.1.1...n)))\-\-\\
    %                 (NÓ \=1.2\+\\ 
    %                     (NÓ \=1.2.1\+\\ 
    %                         (NÓ 1.2.1...n)))))
    %     \end{tabbing}
    % \end{minipage}
    % \end{flushleft}
    \caption[Formato Árvores Deitadas]{Formato Árvores Deitadas. Adaptado de \citeonline[p~6]{afonso2006arvores}}
    \label{fig:bosque_ad}
\end{figure}
\end{center}
O Bosque segue o formato chamado Árvores Deitadas \citeonline[p~6]{afonso2006arvores}, que se apresenta como na Figura \ref{fig:bosque_ad}. Como descrito em \citeonline{freitas2007biblia}, 
\begin{quote}
    \textquote{A cada frase está associada informação textual, isto é, informação relativa ao extracto a que a frase pertence, o número da frase no Bosque e o texto (frase per se). Cada frase é iniciada por A1 (análise 1 da frase). A mesma árvore pode ter mais do que uma análise distinta que são indicadas por A2, A3, etc. [\ldots]\\
    NÓ RAIZ é o nó mais alto da árvore correspondente à sua raiz, por isso é único, isto é, não existem mais nós ao mesmo nível. Assim, o nó raiz não exibe descontinuidade nem pode estar coordenado.\\
    Todos os outros nós constituintes da árvore (NÓ 1 e nós dependentes e os dependentes dos nós dependentes (NÓ 1.1. ou NÓ 1.2. a NÓ 1.1.1....N ou NÓ 1.2.1....N) estão por isso abaixo da raiz da árvore.}
\end{quote}
\begin{center}
\begin{figure}[!ht]
    \centering
    % \includegraphics{}
    \begin{tabular}{l}
        =>N:art('o' <artd> M P) Os\\
        =H:n('promotor' M P) promotores
    \end{tabular}
    \caption[Exemplo de nós no formato AD]{Exemplo de nós no formato AD. Como descrito em \citeonline{freitas2007biblia}, \textquote{A funçao de \textquote{os} é de dependente de um núcleo nominal (N) que está a sua direita (>), por isso a marcação >N; a forma de \textquote{os} é artigo. Tem-se entao o par de função e forma >N:art. \textquote{promotores}, por sua vez, é o núcleo (H) do sintagma, e sua forma é substantivo/nome (n). O par função e forma é portanto H:n}. Adaptado de \citeonline{freitas2007biblia}}
    \label{fig:bosque_ad_exemplo}
\end{figure}
\end{center}
Cabe notar que cada nó segue o formato (F:f), ou seja, \textbf{F}unção e \textbf{f}orma. Como descrito em \citeonline{freitas2007biblia},
\begin{quote}
    \textquote{A função corresponde à função sintáctica (sujeito, predicador, etc.) que cada constituinte possui em cada oração ou sintagma que compõe a frase. A forma corresponde à estrutura interna dos constituintes, isto é, sintagmas e orações para os nós não terminais, e, para os nós terminais, é usada uma classificação muito próxima das classes de PoS (advérbio, adjectivo, etc.).}
\end{quote}
    
Como podemos ver na Figura \ref{fig:bosque_ad_exemplo}, cada nó é rico em informações morfossintáticas. O tratamento para isto será melhor descrito em \ref{sec:treinando_sp_bosque}. As \textit{tags} utilizadas no Bosque, bem como seu nome, podem ser vistos na Tabela \ref{tab:tab_bosque}, ou no anexo 1 da Bíblia Florestal (\citeonline{freitas2007biblia}). A Bíblia é o manual de anotação, que descreve toda estrutura do Projeto. Caso deseje, uma versão impressa, \citeonline{afonso2006arvores} foi disponibilizada. Ao leitor curioso, recomendamos a sua leitura. 


\subsubsection{CINTIL}
\label{subsec:cintil}
O CINTIL é um \textit{dataset} desenvolvido pela \textit{Natural Language and Speech Group} (NLX-Group\footnote{ \url{http://nlx.di.fc.ul.pt/}}) da Universidade de Lisboa\footnote{\url{https://www.ulisboa.pt}}. Seu objetivo é permitir o estudo linguístico da língua portuguesa, variante europeia.

Por \citeonline[p~1]{narrativeDescriptionCintil}, ele é um corpus de árvores sintáticas de constituência, de textos em português, constituído por 10039 sentenças e 110166 tokens, tirados de diversas fontes de domínios: notícias (8861 sentenças, 101430 tokens), romances (339 sentenças, 3082 tokens). Além disso, há também 779 sentenças (5654 tokens) que são usadas para testes de regressão de gramáticas computacionais que apoiaram a anotação. A Tabela \ref{tab:cintil_tags} demonstra a distribuição do corpus.

\begin{center}
\begin{table}[!ht]
    \centering
    \begin{tabular}{|c|c|c|c|c|}
    \hline
        Sub-corpus & id & Sentences & Tokens & Domain\\
    \hline
        Sentences for regression testing & aTSTS & 779 & 5654 & Test\\
        \hline
        CINTIL-International Corpus of Portuguese & bCINT & 1219 & 13516 & News\\
         & cCINT & 399 & 3082 & Novels\\
         \hline
        CETEMPublico & eCTMP & 7541 & 86905 & News\\
        \hline
        Penn TreeBank (translation) & dPENN & 101 & 1012 & News\\
        \hline
        Total &  & 10039 & 110166 & \\
    \hline
    
    \end{tabular}
    \caption[Distribuição de sentenças e tags pelo CINTIL]{Distribuição de sentenças e tags pelo CINTIL. Adaptado de \citeonline{narrativeDescriptionCintil}}
    \label{tab:cintil_tags}
\end{table}
\end{center}

Foi usada uma classificação semi-automática na criação do \textit{treebank}. Como visto no Iness\footnote{\url{http://clarino.uib.no/iness/page?page-id=port-descr}}(\citeonline{rosen2012open}), num primeiro momento, uma \textit{deep computational grammar} \citeonline{lxgram} é usada para gerar todas as possíveis árvores duma sentença. Na sequência, é feita uma desambiguação manual, onde dois anotadores escolhem a melhor árvore. Em caso de empate, um terceiro especialista servirá de árbitro.

O CINTIL é distribuído em formato XML, e as árvores classificadas tem formato semelhante ao do PTB, fazendo separações usando parênteses, e com classes muito parecidas. A Tabela \ref{tab:tab_cintil} demonstra as \textit{tags} originais, e a frequência de uso no banco.

Os padrões de classificação do CINTIL atual está catalogado no \textit{CINTIL TreeBank handbook} \citeonline{cintil_handbook}.

O CINTIL tem uma versão de 2005, e uma mais atual distribuída em 2012, pelo Metashare\footnote{\url{http://metashare.elda.org/repository/browse/cintil-treebank/2a17d622abcd11e1a404080027e73ea242399e2114844f63896f2f92dd31233e/}}. Existe o site de divulgação oficial, cintil.ul.pt\footnote{\url{http://cintil.ul.pt/pt/cintilwhatsin.html}}, porém ele não é atualizado há algum tempo. Isso se nota pois o \textit{tagset} disponibilizado por ele está desatualizado, sendo referente à versão anterior. O mais atual segue as diretrizes do supracitado \textit{Handbook}.
